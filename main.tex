\documentclass{article}
\usepackage{graphicx} % Required for inserting images
\usepackage{amsmath}
\usepackage{amsfonts}
\usepackage{geometry}
\usepackage{gensymb}
\usepackage{cancel}
\usepackage{float}
\usepackage[separate-uncertainty=true]{siunitx}
\usepackage{svg}
\usepackage{animate}
\usepackage{tabularx}
\usepackage{caption}
\usepackage{subcaption}
\usepackage{biblatex}
\usepackage{booktabs}
\usepackage{hyperref}
\usepackage{array}
\newcolumntype{L}[1]{>{\raggedright\let\newline\\\arraybackslash\hspace{0pt}}m{#1}}
\newcolumntype{C}[1]{>{\centering\let\newline\\\arraybackslash\hspace{0pt}}m{#1}}
\newcolumntype{R}[1]{>{\raggedleft\let\newline\\\arraybackslash\hspace{0pt}}m{#1}}
\addbibresource{mathe.bib}

\newgeometry{vmargin={15mm}, hmargin={30mm,30mm}}

\title{MfPI Skript}
\date{February 2026}

\begin{document}

\maketitle

\tableofcontents

\section{"Vorwort"}

Guten Morgen liebe Mitstudierende :) Hier einmal das Mathe Skript von Suris für MfPI in ordentlich formattiert mit Lernressourcen. Um beizutragen, guckt gerne auf Github vorbei: github-link.
Viel Erfolg beim Lernen!

\section{Notation}

Im Folgenden werden alle Bezeichnungen von Objekten in einem \textcolor{blue}{Blauton} markiert, alle Eigenschaften die diese Objekte haben können in einem \textcolor{teal}{Grünton} und alle Verknüpfungen dieser Objekte in einem \textcolor{red}{Rotton}.\\
Hier zusätzlich ein Überblick über die verwendeten Zeichen:
\begin{table}[H]
\label{sammlung}
\begin{tabular}{|L{3cm}|C{5cm}|C{5cm}|R{3cm}|}
\hline
Zeichen            & Sprich & Bedeutung & Latex-Code \\ \hline
$a \in M$          & a ist Element von M& Das Element a ist in M enthalten      & \verb|\in| \\

$ A \Leftrightarrow B$ & A gilt genau dann wenn (g.d.w.) B gilt & beide Aussagen sind gleichbedeutend/äquivalent & \verb|\Leftrightarrow|       \\

$A \subseteq B$    &    A ist eine Teilmenge von B    &           &            \\

$\subset$          &    A ist eine echte Teilmenge von B    &           &            \\

$a$                 &        &           &            \\

$a$                 &        &           &            \\

$a$                 &        &           &            \\

$a$                 &        &           &            \\

$a$                 &        &           &            \\

$a$                 &        &           &            \\

$a$                 &        &           &            \\

$a$                 &        &           &            \\

$a$                  &        &           &            \\

$a$                  &        &           &            \\ \hline
\end{tabular}
\end{table}

\section{Überblick MfPI}

Grundlagen - Mengen, Zahlen
Lineare Algebra
Analysis

\section{Kapitel 1: Grundlagen - Mengen}

\subsection{Grundlegende Definitionen}

\textbf{Definition:} (nach Georg Cantor 1895)\\
Eine \textcolor{blue}{Menge} ist eine Zusammenfassung von bestimmen wohl-unterschiedenen Objekten unserer Anschauung oder unseres Denkens zu einem Ganzen. Diese Objekte heißen \textcolor{blue}{Elemente} der Menge.\\\\
Wir schreiben: $a \in M$ sprich "a ist Element von M" bzw. "M enthält a"\\
Oder gegenteilig: $a \notin M$ sprich "a ist nicht Element von M" bzw. "M enthält a nicht"\\\\
Wie kann man eine \textcolor{blue}{Menge} definieren bzw. beschreiben? Eine \textcolor{blue}{Menge} wird definiert durch die \textcolor{blue}{Elemente} die sie enthält. \textcolor{blue}{Mengen} kann man per Aufzählung beschreiben:\\
$$ M_1 = \{1,2,3\} \quad M_2 = \{2,4,8\} $$
$$ M_1 \neq M_2 $$
Es gibt \textcolor{red}{endliche} \textcolor{blue}{Mengen}:
$$ M_{endlich} = \{1,2,3\} $$
Und \textcolor{red}{unendliche} \textcolor{blue}{Mengen}:
$$ M_{unendlich} = \{1,2,3,4,5,...\} $$
\textcolor{blue}{Mengen} sind in der Regel \textcolor{red}{ungeordnet}, das heißt die Reihenfolge ihrer \textcolor{blue}{Elemente} spielt keine Bedeutung:
$$ M = \{1,2,3\} = \{2,3,1\}$$
\textcolor{blue}{Mengen} können durch Aufzählung beschrieben werden, aber auch durch eine Regel oder Charakteristische Eigenschaft, die alle \textcolor{blue}{Elemente} der \textcolor{blue}{Menge} erfüllen und alle \textcolor{blue}{nicht-Elemente} nicht erfüllen:
$$ M = \{ x : x \text{ hat Eigenschaft E}\} \text{ sprich: "die Menge aller Elemente x, für die E gilt"} $$
Zum Beispiel:
$$ M = \{ x : x \text{ ist eine gerade natürliche Zahl}\} = \{2,4,6,8,...\} $$
Manchmal wird dieser Doppelpunkt auch durch einen Strich $|$ ersetzt.\\
MathematikerInnen verwenden fsr gerne Symbole, deswegen ließe sich das verwendete Beispiel auch so schreiben:
$$ M = \{x : x\in \mathbb{N}; x\%2=0\} $$
Im Skript gibt es jetzt ein Beispiel für ein Paradoxon das auftreten kann, das betrachte wer will, wichtig ist nur es existiert eine:
$$ \exists \emptyset = \{\} = \text{ \textcolor{blue}{Leere Menge}} $$
Diese enthält kein \textcolor{blue}{Element}.\\

\subsection{Teilmengen}
Es gilt:
$$ A \subseteq B \Leftrightarrow \forall x \in A : x \in B$$
Sprich eine Menge $A$ heißt \textcolor{blue}{Teilmenge} von $B$ wenn gilt, dass für jedes \textcolor{blue}{Element} $x$, das es in $A$ gibt, dieses \textcolor{blue}{Element} auch in $B$ vorkommt. Dabei kann $A$ auch gleich $B$ sein ($A=B \Rightarrow A \subseteq B \text{ und } B \subseteq A$). Wenn das nicht der Fall sein darf, so verwenden wir das Symbol ohne den Strich:
$$ A \subset B \Leftrightarrow \forall x \in A : x \in B\text{ und }A\neq B$$
Dann heißt $A$ eine \textcolor{red}{echte} \textcolor{blue}{Teilmenge} von $B$.\\\\
\textbf{Beispielaufgaben:}\\
\begin{itemize}
    \item Ist $M = \{1,2,4,8,16,...\}$ gleich $N = \{x: x = 2^b\}$ mit $b\in \mathbb{N}_0$?
    \item Ist die Menge aller Vögel eine Teilmenge der Menge aller Tiere?
    \item Ist die Menge aller Rationalen Zahlen $\mathbb{Q}$ eine echte Teilmenge von der Menge $ M = \{ x : x = \frac{a}{b}\} $ mit $a,b \in \mathbb{Z}$ ?
\end{itemize}
Bemerke, dass wir auch \textcolor{teal}{Teilmenge} schreiben könnten, da der Begriff eine Relation beschreibt (Eine Menge kann keine Teilmenge an sich sein, sondern nur in Bezug auf eine andere Menge).\\
\textbf{Videoempfehlungen:}
\begin{itemize}
    \item https://studyflix.de/mathematik/mathematische-symbole-5074
    \item https://studyflix.de/mathematik/mengenlehre-3541
\end{itemize}

\subsection{Operationen mit Mengen}



\end{document}
